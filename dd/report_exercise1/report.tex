\documentclass[10pt,a4paper,titlepage,oneside]{article}
\usepackage{LabProtocol}

\exercise{Exercise I}

% enter your data here
\authors{
	Vorname Nachname, Matr. Nr. 0123456 \par
	{\small e0123456@student.tuwien.ac.at} \par
}


\begin{document}

\maketitle


%████████╗ █████╗ ███████╗██╗  ██╗     ██╗
%╚══██╔══╝██╔══██╗██╔════╝██║ ██╔╝    ███║
%   ██║   ███████║███████╗█████╔╝     ╚██║
%   ██║   ██╔══██║╚════██║██╔═██╗      ██║
%   ██║   ██║  ██║███████║██║  ██╗     ██║
%   ╚═╝   ╚═╝  ╚═╝╚══════╝╚═╝  ╚═╝     ╚═╝
\Task{Structural modeling}

\begin{qa}{Create a screenshot showing the top level design in the RTL netlist viewer!}
	\begin{figure}[h!]
		\centering
		% \includegraphics[width=1.0\linewidth]{your filename here}
		\dummyimage
		\caption{RTL netlist viewer screenshot}
	\end{figure}
\end{qa}
%%%%%%%%%%%%%%%%%%%%%%%%%%%%%%%%%%%%%%%%%%%%%%%%%%%%%%%%%%%%%%%%%%%%%%%%%%%%%%%%


%████████╗ █████╗ ███████╗██╗  ██╗    ██████╗ 
%╚══██╔══╝██╔══██╗██╔════╝██║ ██╔╝    ╚════██╗
%   ██║   ███████║███████╗█████╔╝      █████╔╝
%   ██║   ██╔══██║╚════██║██╔═██╗     ██╔═══╝ 
%   ██║   ██║  ██║███████║██║  ██╗    ███████╗
%   ╚═╝   ╚═╝  ╚═╝╚══════╝╚═╝  ╚═╝    ╚══════╝
\Task{Seven Segment Display I}

\begin{qa}{Are the \textsf{hex*} signals high or low-active? Explain what that means!}
put your answer here ...
\end{qa}
%%%%%%%%%%%%%%%%%%%%%%%%%%%%%%%%%%%%%%%%%%%%%%%%%%%%%%%%%%%%%%%%%%%%%%%%%%%%%%%%

%████████╗ █████╗ ███████╗██╗  ██╗    ██████╗ 
%╚══██╔══╝██╔══██╗██╔════╝██║ ██╔╝    ╚════██╗
%   ██║   ███████║███████╗█████╔╝      █████╔╝
%   ██║   ██╔══██║╚════██║██╔═██╗      ╚═══██╗
%   ██║   ██║  ██║███████║██║  ██╗    ██████╔╝
%   ╚═╝   ╚═╝  ╚═╝╚══════╝╚═╝  ╚═╝    ╚═════╝ 
\Task{Behavioral Simulation}

\begin{qa}{SRAM write access}

\begin{center}
\begin{tabular}{lc}
	\hline
	Question & Answer \\
	\hline\hline
	Absoulte (simulation) time of the first write access$^{*}$ & ?? ns \\
	Accessed SRAM addresses (first 4 write operations)         & 0x??, 0x??, 0x??, 0x?? \\
	Measure the SRAM timing parameter $t_{WC}$ (see datasheet) & ?? ns\\\hline
\end{tabular}

\footnotesize{$^{*}$ Take the point in time where the address for the write operation is applied.}
\end{center}

\begin{figure}[h!]
	\centering
	%\includegraphics[width=1.0\linewidth]{path}
	\dummyimage
	\caption{Simulation showing the first 4 write operations to the SRAM with markers indicating the SRAM timing parameter $t_{WC}$}
\end{figure}

\end{qa}
%%%%%%%%%%%%%%%%%%%%%%%%%%%%%%%%%%%%%%%%%%%%%%%%%%%%%%%%%%%%%%%%%%%%%%%%%%%%%%%%

\begin{qa}{Serial port interface of the LCD driver IC (Consult the LCD manual to answer the following questions)}

\begin{figure}[h!]
	\centering
	%\includegraphics[width=1.0\linewidth]{path}
	\dummyimage
	\caption{Screenshot showing the data transmission on the serial port interface of the LCD driver IC with markers indicating the period of the $sclk$ signal.}
\end{figure}

\begin{center}
\begin{tabular}{lc}
	\hline
	Question                                               & Answer \\\hline\hline
	What frequency did you measure for the $sclk$ signal?  & ?? MHz \\
	What is the maximum allowed frequency?                 & ?? MHz \\\hline
\end{tabular}
\end{center}

Hint: Should your simulation show ``zero-width spikes'' on some signal traces, you can simply ignore them.
\end{qa}
%%%%%%%%%%%%%%%%%%%%%%%%%%%%%%%%%%%%%%%%%%%%%%%%%%%%%%%%%%%%%%%%%%%%%%%%%%%%%%%%


\begin{qa}{What is the purpose of the transmission on the serial interface to the LCD? Explain what is transmitted and why!}

\end{qa}
%%%%%%%%%%%%%%%%%%%%%%%%%%%%%%%%%%%%%%%%%%%%%%%%%%%%%%%%%%%%%%%%%%%%%%%%%%%%%%%%

%████████╗ █████╗ ███████╗██╗  ██╗    ██╗  ██╗
%╚══██╔══╝██╔══██╗██╔════╝██║ ██╔╝    ██║  ██║
%   ██║   ███████║███████╗█████╔╝     ███████║
%   ██║   ██╔══██║╚════██║██╔═██╗     ╚════██║
%   ██║   ██║  ██║███████║██║  ██╗         ██║
%   ╚═╝   ╚═╝  ╚═╝╚══════╝╚═╝  ╚═╝         ╚═╝
\Task{Postlayout Simulation}
\begin{qa}{Use a postlayout simualtion to measure the different delays on the \textsf{hex\{6-7\}} signals.}

\begin{center}
\begin{tabular}{lc}
	\hline
	Interval                                                  & Time \\ \hline\hline
	Duration between the first and the last bit toggling      &  ?? \\
	Time between the last active clock edge and stabilization &  ?? \\\hline
\end{tabular}
\end{center}

\begin{figure}[h!]
	\centering
	%\includegraphics[width=1.0\linewidth]{path}
	\dummyimage
	\caption{Screenshot showing the interval measurements with markers}
\end{figure}

\end{qa}
%%%%%%%%%%%%%%%%%%%%%%%%%%%%%%%%%%%%%%%%%%%%%%%%%%%%%%%%%%%%%%%%%%%%%%%%%%%%%%%%



%████████╗ █████╗ ███████╗██╗  ██╗    ███████╗
%╚══██╔══╝██╔══██╗██╔════╝██║ ██╔╝    ██╔════╝
%   ██║   ███████║███████╗█████╔╝     ███████╗
%   ██║   ██╔══██║╚════██║██╔═██╗     ╚════██║
%   ██║   ██║  ██║███████║██║  ██╗    ███████║
%   ╚═╝   ╚═╝  ╚═╝╚══════╝╚═╝  ╚═╝    ╚══════╝
\Task{Testbench Design}

\begin{qa}{PRNG simulation}

\begin{center}
\begin{tabular}{ll}
\hline\hline
My matriculation number             & ?? \\
My matriculation number modulo $15$ & ?? \\
$n_a$                               & ?? \\
$n_b$                               & ?? \\
Minimum period                      & ?? \\
Maximum period                      & ?? \\\hline
\end{tabular}
\end{center}

\end{qa}
%%%%%%%%%%%%%%%%%%%%%%%%%%%%%%%%%%%%%%%%%%%%%%%%%%%%%%%%%%%%%%%%%%%%%%%%%%%%%%%%

\begin{qa}{Bonus Task: SRAM reads}
\begin{figure}[h!]
	\centering
	%\includegraphics[width=1.0\linewidth]{path}
	\dummyimage
	\caption{Simulation showing a read operation performed by the \textsf{lcd\_graphics\_controller}}
\end{figure}
\end{qa}
%%%%%%%%%%%%%%%%%%%%%%%%%%%%%%%%%%%%%%%%%%%%%%%%%%%%%%%%%%%%%%%%%%%%%%%%%%%%%%%%



%████████╗ █████╗ ███████╗██╗  ██╗     ██████╗ 
%╚══██╔══╝██╔══██╗██╔════╝██║ ██╔╝    ██╔════╝ 
%   ██║   ███████║███████╗█████╔╝     ███████╗ 
%   ██║   ██╔══██║╚════██║██╔═██╗     ██╔═══██╗
%   ██║   ██║  ██║███████║██║  ██╗    ╚██████╔╝
%   ╚═╝   ╚═╝  ╚═╝╚══════╝╚═╝  ╚═╝     ╚═════╝ 
\Task{N64 Controller}

\begin{qa}{Simulation of two complete button state transmission frames}

\begin{center}
\begin{tabular}{ll}
\hline\hline
My matriculation number                   &  ?? \\
My matriculation number modulo $2^{16}$   & 0x?? \\
$b_0$: hex, ({A},{B},{Select},{Start},{$\uparrow$},{$\downarrow$},{$\leftarrow$},{$\rightarrow$})  & 0x??, (?,?,?,?,?,?,?,?) \\
$b_1$: hex, ({A},{B},{Select},{Start},{$\uparrow$},{$\downarrow$},{$\leftarrow$},{$\rightarrow$})  & 0x??, (?,?,?,?,?,?,?,?) \\\hline
\end{tabular}
\end{center}

\begin{figure}[h!]
	\centering
	\dummyimage
	%\includegraphics[width=1.0\linewidth]{path}
	\caption{Screenshot(s) showing the both transmissions with markers shing the clock period of $nes\_clk$}
\end{figure}

\end{qa}
%%%%%%%%%%%%%%%%%%%%%%%%%%%%%%%%%%%%%%%%%%%%%%%%%%%%%%%%%%%%%%%%%%%%%%%%%%%%%%%%


\begin{qa}{Analyse the resource usage of your \textsf{nes\_controller}!}
\centering
\begin{tabular}{l|ll}
	\hline
		                   & LC Combinationals & LC Registers        \\ \hline\hline 
	Absolute number            &                   &                     \\
	\% of whole design         &                   &                     \\ 
	\% of whole FPGA resources &                   &                     \\ \hline
\end{tabular}
\end{qa}
%%%%%%%%%%%%%%%%%%%%%%%%%%%%%%%%%%%%%%%%%%%%%%%%%%%%%%%%%%%%%%%%%%%%%%%%%%%%%%%%


%████████╗ █████╗ ███████╗██╗  ██╗    ███████╗
%╚══██╔══╝██╔══██╗██╔════╝██║ ██╔╝    ╚════██║
%   ██║   ███████║███████╗█████╔╝         ██╔╝
%   ██║   ██╔══██║╚════██║██╔═██╗        ██╔╝ 
%   ██║   ██║  ██║███████║██║  ██╗       ██║  
%   ╚═╝   ╚═╝  ╚═╝╚══════╝╚═╝  ╚═╝       ╚═╝  
\Task{Seven Segment Display II}

\begin{qa}{Include the state graph of the state machine you designed and briefly explain how it works.}

You can use \texttt{dia} to draw the diagram. The provided makefile automatically converts dia files to PDFs and places them in the \texttt{dia/pdf} directory.
However, any other method for drawing pictures is also fine. 

\begin{figure}[h!]
	\centering
	\includegraphics[width=0.75\linewidth]{dia/pdf/example_fsm.pdf}
	\caption{FSM state graph}
\end{figure}

\end{qa}
%%%%%%%%%%%%%%%%%%%%%%%%%%%%%%%%%%%%%%%%%%%%%%%%%%%%%%%%%%%%%%%%%%%%%%%%%%%%%%%%


\begin{qa}{Seven Segment Display Simulation Screenshot}

\begin{center}
\begin{tabular}{ll}
\hline\hline
My matriculation number                          & ?? \\ 
My matriculation number modulo $500$             & ?? \\ 
\textsf{player\_points} value for the testbench  & ?? \\ \hline
\end{tabular}
\end{center}

\begin{figure}[h!]
	\centering
	%\includegraphics[width=1.0\linewidth]{path}
	\dummyimage
	\caption{Simulation showing the conversion}
\end{figure}

\end{qa}
%%%%%%%%%%%%%%%%%%%%%%%%%%%%%%%%%%%%%%%%%%%%%%%%%%%%%%%%%%%%%%%%%%%%%%%%%%%%%%%%

\begin{qa}{Bonus Task: Animation Simulation}
\begin{figure}[h!]
	\centering
	%\includegraphics[width=1.0\linewidth]{path}
	\dummyimage
	\caption{Simulation showing all animation steps}
\end{figure}
\end{qa}
%%%%%%%%%%%%%%%%%%%%%%%%%%%%%%%%%%%%%%%%%%%%%%%%%%%%%%%%%%%%%%%%%%%%%%%%%%%%%%%%


\end{document}
